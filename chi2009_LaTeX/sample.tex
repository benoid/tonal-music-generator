\documentclass{chi2009}
\usepackage{times}
\usepackage{url}
\usepackage{graphics}
\usepackage{color}
\usepackage[pdftex]{hyperref}
\hypersetup{%
pdftitle={Your Title},
pdfauthor={Your Authors},
pdfkeywords={your keywords},
bookmarksnumbered,
pdfstartview={FitH},
colorlinks,
citecolor=black,
filecolor=black,
linkcolor=black,
urlcolor=black,
breaklinks=true,
}
\newcommand{\comment}[1]{}
\definecolor{Orange}{rgb}{1,0.5,0}
\newcommand{\todo}[1]{\textsf{\textbf{\textcolor{Orange}{[[#1]]}}}}

\pagenumbering{arabic}  % Arabic page numbers for submission.  Remove this line to eliminate page numbers for the camera ready copy

\begin{document}
% to make various LaTeX processors do the right thing with page size
\special{papersize=8.5in,11in}
\setlength{\paperheight}{11in}
\setlength{\paperwidth}{8.5in}
\setlength{\pdfpageheight}{\paperheight}
\setlength{\pdfpagewidth}{\paperwidth}

% use this command to override the default ACM copyright statement 
% (e.g. for preprints). Remove for camera ready copy.
\toappear{Submitted for review to CHI 2009.}

\title{Markov Decision in Tonal Music Generation}
\numberofauthors{2}
\author{
  \alignauthor David Benoit\\
    \affaddr{University of Massachusetts Lowell}\\
    \affaddr{Department of Computer Science}\\
    \email{david\_benoit@student.uml.edu}
  \alignauthor Brian Chiang\\
    \affaddr{University of Massachusetts Lowell}\\
    \affaddr{Department of Computer Science}\\
    \email{brian\_chiang@student.uml.edu}
}

\maketitle

\begin{abstract}
The objective of this project was to explore applications of 
Markov Decision Processes to the generation of tonal music.  
\end{abstract}

\keywords{put author keywords here} 

\category{H.5.2}{Information Interfaces and Presentation}{Miscellaneous}[Optional sub-category]

\section{Introduction}

Markov decision processes, or MDPs, have been used frequently to generate 
tonal music.  Markov decision processes revolve around the idea that 
at any given point in time, the process is at a state, s, and may choose
an action, a, to move the process to a new state, s’.  
Notable variants in MDPs are value iteration and policy iteration. 

\section{Project Description}

\section{Code}

\section{Analysis of Results}

\section{Discussion}

\section{Conclusion}

\section{Acknowledgments}

\section{References}

\section{Page Size and Columns}

On each page your material (not including the page number) should fit
within a rectangle of 18 x 23.5 cm (7 x 9.25 in.), centered on a US
letter page, beginning 1.9 cm (.75 in.) from the top of the page, with
a .85 cm (.33 in.) space between two 8.4 cm (3.3 in.) columns.  On an
A4 page, use a text area of the same dimensions (18 x 23.5 cm.), again
centered.  Right margins should be justified, not ragged. Beware,
especially when using this template on a Macintosh, Word can change
these dimensions in unexpected ways.

\section{Typeset Text}

Prepare your submissions on a word processor or typesetter.  Please
note that page layout may change slightly depending upon the printer
you have specified.  For this document, printing to Adobe Acrobat PDF
Writer was specified.  In the resulting page layout, Figure 1 appears
at the top of the left column on page 2, and Table 1 appears at the
top of the right column on page 2.  You may need to reposition the
figures if your page layout or PDF-generation software is different.

\subsection{Title and Authors}

Your paper's title, authors and affiliations should run across the
full width of the page in a single column 17.8 cm (7 in.) wide.  The
title should be in Helvetica 18-point bold; use Arial if Helvetica is
not available.  Authors' names should be in Times Roman 12-point bold,
and affiliations in Times Roman 12-point (note that Author and
Affiliation are defined Styles in this template file).

To position names and addresses, use a single-row table with invisible
borders, as in this document.  Alternatively, if only one address is
needed, use a centered tab stop to center all name and address text on
the page; for two addresses, use two centered tab stops, and so
on. For more than three authors, you may have to place some address
information in a footnote, or in a named section at the end of your
paper. Please use full international addresses and telephone dialing
prefixes.  Leave one 10-pt line of white space below the last line of
affiliations.

\subsection{Abstract and Keywords}

Every submission should begin with an abstract of about 150 words,
followed by a set of keywords. The abstract and keywords should be
placed in the left column of the first page under the left half of the
title. The abstract should be a concise statement of the problem,
approach and conclusions of the work described.  It should clearly
state the paper's contribution to the field of HCI.

The first set of keywords will be used to index the paper in the
proceedings. The second set are used to catalogue the paper in the ACM
Digital Library. The latter are entries from the ACM Classification
System~\cite{acm_categories}.  In general, it should only be necessary
to pick one or more of the H5 subcategories, see
http://www.acm.org/class/1998/H.5.html

\subsection{Normal or Body Text}

Please use a 10-point Times Roman font or, if this is unavailable,
another proportional font with serifs, as close as possible in
appearance to Times Roman 10-point. The Press 10-point font available
to users of Script is a good substitute for Times Roman. If Times
Roman is not available, try the font named Computer Modern Roman. On a
Macintosh, use the font named Times and not Times New Roman. Please
use sans-serif or non-proportional fonts only for special purposes,
such as headings or source code text.

\subsection{First Page Copyright Notice}

Leave 3 cm (1.25 in.) of blank space for the copyright notice at the
bottom of the left column of the first page. In this template a
floating text box will automatically generate the required space.

\subsection{Subsequent Pages}

On pages beyond the first, start at the top of the page and continue
in double-column format.  The two columns on the last page should be
of equal length.

\subsection{References and Citations}

Use a numbered list of references at the end of the article, ordered
alphabetically by first author, and referenced by numbers in brackets
[2,4,5,7]. For papers from conference proceedings, include the title
of the paper and an abbreviated name of the conference (e.g., for
Interact 2003 proceedings, use Proc. Interact 2003). Do not include
the location of the conference or the exact date; do include the page
numbers if available. See the examples of citations at the end of this
document. Within this template file, use the References style for the
text of your citation.

Your references should be published materials accessible to the
public.  Internal technical reports may be cited only if they are
easily accessible (i.e., you provide the address for obtaining the
report within your citation) and may be obtained by any reader for a
nominal fee.  Proprietary information may not be cited. Private
communications should be acknowledged in the main text, not referenced
(e.g., ``[Robertson, personal communication]'').


\bibliographystyle{abbrv}
\bibliography{sample}

\end{document}
